\chapter{Конструкторский раздел}
\section{Диаграмма проектируемой базы данных}
На рисунке~\ref{fig:BD} представлена диаграмма  проектируемой базы данных.

\begin{figure}
	\centering
	\includegraphics[width=1.5\linewidth, angle=-90]{pictures/BD_2}
	\caption{Диаграмма проектируемой базы данных}
	\label{fig:BD}
\end{figure}

\section{Описание сущностей проектируемой базы данных}
\begin{itemize}
	\item \textbf{Списание}
	\begin{itemize}
		\item Основная сущность для учёта списания товаров
		\item Атрибуты:
		\begin{itemize}
			\item id - уникальный идентификатор
			\item номер партии
			\item номенклатура
			\item количество списываемого товара
			\item ответственный за списание (пользователь с ролью кладовщик)
		\end{itemize}
	\end{itemize}
	
	\item \textbf{Пользователи}
	\begin{itemize}
		\item Сущность для хранения данных пользователей системы
		\item Атрибуты:
		\begin{itemize}
			\item id - уникальный идентификатор
			\item название (имя пользователя)
			\item роль в системе
		\end{itemize}
	\end{itemize}
	
	\item \textbf{Страна}
	\begin{itemize}
		\item Справочник стран
		\item Атрибуты:
		\begin{itemize}
			\item id - уникальный идентификатор
			\item название страны
		\end{itemize}
	\end{itemize}
	
	\item \textbf{Товар на складе}
	\begin{itemize}
		\item Сущность учёта текущих остатков на складе
		\item Атрибуты:
		\begin{itemize}
			\item id - уникальный идентификатор
			\item номенклатура
			\item количество товара
			\item место хранения
			\item номер партии
		\end{itemize}
	\end{itemize}
	
	\item \textbf{Заказ}
	\begin{itemize}
		\item Сущность для оформления заказов
		\item Атрибуты:
		\begin{itemize}
			\item id - уникальный идентификатор
			\item дата
			\item контрагент
			\item ответственный за заказ
		\end{itemize}
	\end{itemize}
	
	\item \textbf{Заказ товар}
	\begin{itemize}
		\item Сущность для сопоставления заказа и товаров
		\item Атрибуты:
		\begin{itemize}
			\item id заказа
			\item id товара
			\item количество
			\item прайс
		\end{itemize}
	\end{itemize}
	
	\item \textbf{Контрагент}
	\begin{itemize}
		\item Сущность для хранения данных контрагентов
		\item Атрибуты:
		\begin{itemize}
			\item id - уникальный идентификатор
			\item название
			\item тип (поставщик/покупатель)
			\item юридический статус
			\item юридический адрес
			\item контактное лицо
			\item телефон
		\end{itemize}
	\end{itemize}
	
	\item \textbf{Продажи}
	\begin{itemize}
		\item Сущность для учёта продаж
		\item Атрибуты:
		\begin{itemize}
			\item номер чека
			\item контрагент
			\item заказ
			\item статус продажи
			\item итоговая цена
		\end{itemize}
	\end{itemize}
	
	\item \textbf{Места хранения}
	\begin{itemize}
		\item Справочник складских помещений
		\item Атрибуты:
		\begin{itemize}
			\item id - уникальный идентификатор
			\item название места хранения
			\item адрес
		\end{itemize}
	\end{itemize}
	
	\item \textbf{Номенклатура}
	\begin{itemize}
		\item Основной справочник товаров/продукции
		\item Атрибуты:
		\begin{itemize}
			\item id - уникальный идентификатор
			\item название товара
			\item страна производства
		\end{itemize}
	\end{itemize}
	
	\item \textbf{Партия}
	\begin{itemize}
		\item Учёт поступлений товаров партиями
		\item Атрибуты:
		\begin{itemize}
			\item id - уникальный идентификатор
			\item дата
			\item дата годности
			\item себестоимость
			\item количество
			\item ответственный
			\item поставщик
		\end{itemize}
	\end{itemize}
	
	\item \textbf{Прайс}
	\begin{itemize}
		\item Справочник цен на товары
		\item Атрибуты:
		\begin{itemize}
			\item id - уникальный идентификатор
			\item номенклатура
			\item цена
			\item партия
		\end{itemize}
	\end{itemize}
	
\end{itemize}
\section{Описание проектируемых ограничений целостности базы данных}
до сих пор не понимаю, что тут должно быть
\section{Описание всех проектируемых процедур/функций/триггеров в формате схемы}
На рисунке~\ref{fig:trigger1} представлена диаграмма алгоритма триггера на обновление товара на складе при поступлении новой партии товара.
\begin{figure}
	\centering
	\includegraphics[width=1\linewidth]{pictures/trigger_1_load_batch}
	\caption{Диаграмма алгоритма триггера на обновление товара на складе при поступлении новой партии товара}
	\label{fig:trigger1}
\end{figure}

На рисунке~\ref{fig:trigger2} представлена диаграмма алгоритма триггера на обновление товара на складе при покупке.
\begin{figure}
	\centering
	\includegraphics[width=1\linewidth]{pictures/trigger_2_make_purchase}
	\caption{Диаграмма алгоритма триггера на обновление товара на складе при покупке}
	\label{fig:trigger2}
\end{figure}
\section{Описание проектируемой ролевой модели на уровне базы данных}
Три имеющиеся типа пользователей обладают следующим набором прав. Кладовщики -- загрузка информации  о партии и просмотр каталога доступных товаров; продавцы -- внесение информации о заказах и продажах, просмотр каталога доступных товаров; администраторы -- всё выше перечисленное. 
\section*{Вывод}
В конструкторском разделе была разработана структура базы данных, включающая 12 нормализованных сущностей, связанных между собой через внешние ключи, что обеспечивает целостность данных. Диаграмма базы данных наглядно демонстрирует взаимосвязи между таблицами: от справочников (номенклатура, страны, места хранения) до операционных сущностей (партии, заказы, продажи, списания). Особое внимание уделено механизмам поддержания актуальности данных через систему триггеров, автоматически обновляющих остатки на складе при поступлении новых партий и продажах. Ролевая модель, реализованная на уровне СУБД, разделяет права доступа между тремя типами пользователей (администраторами, продавцами и кладовщиками), обеспечивая безопасность и соответствие бизнес-процессам компании.