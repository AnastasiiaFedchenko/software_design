\chapter{Технологический раздел}
\section{Обоснование выбора средств реализации базы данных и приложения}
Выбор PostgreSQL обусловлен его надежностью, производительностью и соответствием требованиям предметной области. Система предоставляет: полноценную поддержку ACID-транзакций для гарантии целостности данных, гибкую систему ограничений и триггеров для автоматизации бизнес-процессов, встроенную ролевую модель с детализированным управлением доступом, поддержку сложных типов данных (JSONB, ENUM), что упрощает хранение структурированной информации.

Для клиентской части выбор пал на $C\#$ в силу: строгой типизации и высокой производительности исполняемого кода, наличия проверенных инструментов интеграции с PostgreSQL (Npgsql), поддержки современных парадигм программирования и паттернов проектирования.

Данный технологический стек оптимально соответствует требованиям к безопасности, надежности и масштабируемости проектируемой системы учета.
\section{Описание сущностей реализованной базы данных}
ещё раз написать то же самое?
\section{Описание реализованных ограничений целостности базы данных}
тоже не понятно
\section{Описание всех реализованных процедур/функций/триггеров в формате схемы}
вроде бы ровно такой же пункт есть в конструкторской части
\section{Описание ролевой модели на уровне базы данных}
прям точно какие таблицы могут insert update delete?
\section{Описание методов тестирования и тестовых кейсов для всех разработанных на стороне базы данных функций}
может ли это быть ручное тестирование?
\section{Описание интерфейса доступа к базе данных}
описание интерфейса программы?
