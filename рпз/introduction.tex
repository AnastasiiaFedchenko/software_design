\chapter*{ВВЕДЕНИЕ}
\addcontentsline{toc}{chapter}{ВВЕДЕНИЕ}

Данная работа посвящена проектированию и реализации базы данных и приложения для автоматизации работы цветочного магазина. Основное внимание уделяется разработке структуры базы данных, обеспечению целостности хранимых данных, а также созданию функционального приложения, позволяющего эффективно управлять товарными запасами, учетными записями пользователей и бизнес-процессами магазина.

\textbf{Цель:} создание надежной и масштабируемой базы данных и приложения для цветочного магазина, обеспечивающих автоматизацию ключевых операций.

\textbf{Задачи:}
\begin{enumerate}[label={\arabic*)}]
	\item сформулировать требования и ограничения к разрабатываемой базе данных и приложению;
	\item определить роли и описание пользователей системы;
	\item спроектировать сущности базы данных и ограничения целостности;
	\item реализовать триггер для автоматического обновления товарных остатков при поступлении новых партий;
	\item выбрать средства реализации базы данных и приложения;
	\item разработать и реализовать сущности базы данных с учетом ограничений целостности;
	\item описать методы тестирования функционала и разработать тестовые сценарии;
	\item провести исследование производительности запросов в зависимости от объема данных с использованием индексов и без них.
\end{enumerate}
