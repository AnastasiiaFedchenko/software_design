\chapter*{ПРИЛОЖЕНИЕ Б}
\addcontentsline{toc}{chapter}{ПРИЛОЖЕНИЕ Б}

\begin{center}
	\textbf{Реализации алгоритмов}
\end{center}

\begin{lstlisting}[label=lst:bubble_positioning, caption=Реализация алгоритма расстановки пузырьков]
	public static List<Obj> PositionBubbles(Bubble b1, Bubble b2, bool from_combined)
	{
		double contactAngle = ContactAngle(b1, b2);
		List<Obj> res = new List<Obj>();
		if (contactAngle.Equals(double.NaN))
		return null;
		if (contactAngle < 55.0)
		{
			Bubble res_b = MergeBubbles(b1, b2);
			res.Add(res_b);
		}
		else if (contactAngle > 65.0)
		{
			List<Bubble> bubbles = PushBubblesApart(b1, b2, from_combined);
			res.Add(bubbles[0]);
			res.Add(bubbles[1]);
		}
		else
		{
			if (from_combined)
			return null;
			CombinedBubble cluster = new CombinedBubble(b1.Id, b1, b2);
			cluster.CreateClusters();
			res.Add(cluster);
		}
		return res;
	}
\end{lstlisting}

\begin{lstlisting}[label=lst:merge_bubbles, caption=Реализация слияния в один большой пузырь]
	private static Bubble MergeBubbles(Bubble b1, Bubble b2)
	{
		if (b1 == null || b2 == null)
		return null;
		
		Vector3D newCenter = (b1.Center + b2.Center) / 2;
		
		double volume1 = (4.0 / 3.0) * Math.PI * Math.Pow(b1.Radius, 3);
		double volume2 = (4.0 / 3.0) * Math.PI * Math.Pow(b2.Radius, 3);
		
		double totalVolume = volume1 + volume2;
		
		double newRadius = Math.Pow((totalVolume * 3.0) / (4.0 * Math.PI), 1.0 / 3.0);
		
		b1.Center = newCenter;
		b1.Radius = newRadius;
		
		b2 = null;
		
		return b1;
	}
\end{lstlisting}

\begin{lstlisting}[label=lst:push_bubbles_apart, caption=Реализация отталкивания пузырьков друг от друга]
	private static List<Bubble> PushBubblesApart(Bubble b1, Bubble b2, bool from_combined)
	{
		List<Bubble> res = new List<Bubble>();
		if (b1 == null || b2 == null)
		{
			res.Add(b1);
			res.Add(b2);
			return res;
		}
		
		Vector3D direction = b2.Center - b1.Center;
		double currentDistance = direction.Length;
		double requiredDistance = b1.Radius + b2.Radius;
		
		if (currentDistance < requiredDistance)
		{
			direction.Normalize();
			
			double pushDistance = requiredDistance - currentDistance;
			if (from_combined)
			b2.Center += direction * (pushDistance + 0.0000002);
			else
			{
				b1.Center -= (direction * (pushDistance / 2 + 0.0000001));
				b2.Center += direction * (pushDistance / 2 + 0.0000001); 
			}
			b1.Radius = b1.Radius;
			b2.Radius = b2.Radius;
		}
		res.Add(b1);
		res.Add(b2);
		return res;
	}
\end{lstlisting}

\begin{lstlisting}[label=lst:create_bubble_cluster, caption=Реализация создания пузырькового кластера]
	private void CreateClusters()
	{
		if (bubble1 == null || bubble2 == null)
		return;
		
		List<Vector3D> Points = GetIntersectionPoint(bubble1, bubble2);
		Vector3D centerOfIntersect = Points[0];
		
		Vector3D membraneNormal = bubble2.Center - bubble1.Center;
		membraneNormal.Normalize();
		
		bubble1.MembraneNormal = membraneNormal;
		bubble2.MembraneNormal = membraneNormal;
		
		SphericalSegment newPart1A = CreateSegment(bubble1.Part1, bubble1.MembraneNormal);
		SphericalSegment newPart1B = CreateSegment(bubble1.Part2, bubble1.MembraneNormal);
		
		SphericalSegment newPart2A = CreateSegment(bubble2.Part1, bubble2.MembraneNormal);
		SphericalSegment newPart2B = CreateSegment(bubble2.Part2, bubble2.MembraneNormal);
		
		newPart1A.Direction = membraneNormal;
		newPart1B.Direction = -newPart1A.Direction;
		
		newPart2A.Direction = -newPart1A.Direction;
		newPart2B.Direction = -newPart1B.Direction;
		
		newPart1A.Height = CalculateHeight(newPart1A, centerOfIntersect, newPart1A.Direction);
		newPart1B.Height = CalculateHeight(newPart1B, centerOfIntersect, newPart1B.Direction);
		newPart2A.Height = CalculateHeight(newPart2A, centerOfIntersect, newPart2A.Direction);
		newPart2B.Height = CalculateHeight(newPart2B, centerOfIntersect, newPart2B.Direction);
		
		bubble1.Part1 = newPart1A;
		bubble1.Part2 = newPart1B;
		bubble2.Part1 = newPart2A;
		bubble2.Part2 = newPart2B;
		
		double membraneRadius = bubble1.MembraneRadius;
		bubble1.Membrane = new Circle3D(centerOfIntersect, membraneRadius, bubble1.MembraneNormal);
		bubble2.Membrane = new Circle3D(centerOfIntersect, membraneRadius, bubble2.MembraneNormal);
	}
\end{lstlisting}

\begin{lstlisting}[label=lst:position_objects, caption=Реализация алгоритма анализа объектов сцены]
	private int position_bubbles(int n)
	{
		if (n <= 0)
			return -1;
		bool any_intersection = false;
		bool any_more_than_2 = false;
		List<List<int>> a = counting_contacts(ref any_intersection, ref  any_more_than_2);
		
		if (any_more_than_2)
		{
			MessageBox.Show(
			"More than two contacts",
			"ERROR");
			return -1;
		}
		else if (any_intersection == false)
		{
			Console.WriteLine("no intersections");
			return 0;
		}
		else
		{
			for (int i = 0; i < a.Count; i++)
			for (int j = 0; j < a[i].Count; j++)
			{
				if (a[i][j] == 1)
				{
					int rc = 0;
					if (drawer.Spheres(i) is Bubble && drawer.Spheres(j) is Bubble)
						rc = position_two_bubbles(i, j, (Bubble)drawer.Spheres(i), (Bubble)drawer.Spheres(j));
					else if (drawer.Spheres(i) is CombinedBubble && drawer.Spheres(j) is Bubble)
						rc = position_cluster_and_bubble(i, j, (CombinedBubble)drawer.Spheres(i), (Bubble)drawer.Spheres(j));
					else if (drawer.Spheres(i) is Bubble b4 && drawer.Spheres(j) is CombinedBubble cb2)
						rc = position_cluster_and_bubble(j, i, (CombinedBubble)drawer.Spheres(j), (Bubble)drawer.Spheres(i));
					if (rc == -1)
						return -1;
					a[i][j] = 0;
					a[j][i] = 0;
					position_bubbles(n);
				}
			}
		}
		
		return position_bubbles(n - 1);
	}
\end{lstlisting} 
\begin{lstlisting}[label=lst:trace_ray, caption=Реализация алгоритма обратной трассировки лучей]
	public void Render()
	{
		g.FillRectangle(new SolidBrush(Form1.DefaultBackColor),
		0, 0, canvas_buffer.Width, canvas_buffer.Height);
		
		int w = canvas_buffer.Width / 2;
		int h = canvas_buffer.Height / 2;
		
		for (double x = -w; x < w; x++)
		{
			for (double y = -h; y < h; y++)
			{
				Vector3D direction = CanvasToViewport(w, h, x, y);
				System.Drawing.Color color = TraceRay(new TraceRayArgs((int)x, (int)y, camera_position, direction, 1, double.PositiveInfinity, recursion_depth));
				PutPixel((int)x, (int)y, color, false);
			}
		}
	}
	System.Drawing.Color TraceRay(Object obj)
	{
		if (obj is TraceRayArgs args)
		{
			KeyValuePair<Bubble, double> intersection = ClosestIntersection(args.origin, args.direction, args.min_t, args.max_t);
			if (intersection.Key == null)
				return background_color;
			
			Bubble closest = intersection.Key;
			double closest_t = intersection.Value;
			
			Vector3D point = args.origin + args.direction * closest_t;
			Vector3D normal = point - closest.Center;
			normal.Normalize();
			
			Vector3D view = args.direction * (-1);
			double lighting = ComputeLighting(point, normal, view, closest.Specular);
			System.Drawing.Color local_color = closest.Color_mul(lighting);
			
			if (closest.Reflective <= 0 || args.depth <= 0)
				return local_color;
			
			double koef = 1 - closest.Reflective;
			System.Drawing.Color local_contribution = System.Drawing.Color.FromArgb(
			(int)(local_color.R * koef),
			(int)(local_color.G * koef),
			(int)(local_color.B * koef)
			);
			
			Vector3D reflected_ray = ReflectRay(view, normal);
			System.Drawing.Color reflected_color = TraceRay(new TraceRayArgs(args.x, args.y, point, reflected_ray, EPSILON,
			double.PositiveInfinity, args.depth - 1));
			System.Drawing.Color reflected_contribution = System.Drawing.Color.FromArgb(
			(int)(reflected_color.R * closest.Reflective),
			(int)(reflected_color.G * closest.Reflective),
			(int)(reflected_color.B * closest.Reflective)
			);
			
			System.Drawing.Color result_color = System.Drawing.Color.FromArgb(
			Math.Min(255, local_contribution.R + reflected_contribution.R),
			Math.Min(255, local_contribution.G + reflected_contribution.G),
			Math.Min(255, local_contribution.B + reflected_contribution.B)
			);

			return result_color;
		}
		return background_color;
	}
\end{lstlisting}