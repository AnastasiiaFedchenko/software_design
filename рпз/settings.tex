\usepackage[T2A]{fontenc}
\usepackage[utf8]{inputenc}
\usepackage[english,russian]{babel}
\usepackage{amssymb,amsfonts,amsmath,mathtext,cite,enumerate,float}
\usepackage{pgfplots}
\usepackage{graphicx}
\usepackage{subcaption}
\usepackage{tocloft}
\usepackage{listings}
\usepackage{caption}
\usepackage{tempora}
\usepackage{titlesec}
\usepackage{setspace}
\usepackage{geometry}
\usepackage{indentfirst}
\usepackage{pdfpages}
\usepackage{enumerate,letltxmacro}
\usepackage{threeparttable}
\usepackage{hyperref}
\usepackage{flafter}
\usepackage{enumitem}
\usepackage{multirow}
\usepackage{gensymb}
\usepackage{tabularx}
\usepackage{booktabs}

\usepackage[figure,table]{totalcount}
\usepackage{lastpage}

\setlist{nosep}

\newcommand{\ssr}[1]{\begin{center}
		\LARGE\bfseries{#1}
	\end{center} \addcontentsline{toc}{chapter}{#1}  }

\makeatletter
\renewcommand\LARGE{\@setfontsize\LARGE{22pt}{20}}
\renewcommand\Large{\@setfontsize\Large{20pt}{20}}
\renewcommand\large{\@setfontsize\large{16pt}{20}}
\makeatother
\RequirePackage{titlesec}
\titleformat{\chapter}[block]{\hspace{\parindent}\large\bfseries}{\thechapter}{0.5em}{\large\bfseries\raggedright}
\titleformat{name=\chapter,numberless}[block]{\hspace{\parindent}}{}{0pt}{\large\bfseries\centering}
\titleformat{\section}[block]{\hspace{\parindent}\large\bfseries}{\thesection}{0.5em}{\large\bfseries\raggedright}
\titleformat{\subsection}[block]{\hspace{\parindent}\large\bfseries}{\thesubsection}{0.5em}{\large\bfseries\raggedright}
\titleformat{\subsubsection}[block]{\hspace{\parindent}\large\bfseries}{\thesubsection}{0.5em}{\large\bfseries\raggedright}
\titlespacing{\chapter}{12.5mm}{-22pt}{10pt}
\titlespacing{\section}{12.5mm}{10pt}{10pt}
\titlespacing{\subsection}{12.5mm}{10pt}{10pt}
\titlespacing{\subsubsection}{12.5mm}{10pt}{10pt}

\makeatletter
\renewcommand{\@biblabel}[1]{#1.}
\makeatother
%
%\titleformat{\chapter}[hang]{\LARGE\bfseries}{\hspace{1.25cm}\thechapter}{1ex}{\LARGE\bfseries}
%\titleformat{\section}[hang]{\Large\bfseries}{\hspace{1.25cm}\thesection}{1ex}{\Large\bfseries}
%\titleformat{name=\section,numberless}[hang]{\Large\bfseries}{\hspace{1.25cm}}{0pt}{\Large\bfseries}
%\titleformat{\subsection}[hang]{\large\bfseries}{\hspace{1.25cm}\thesubsection}{1ex}{\large\bfseries}
%\titlespacing{\chapter}{0pt}{-\baselineskip}{\baselineskip}
%\titlespacing*{\section}{0pt}{\baselineskip}{\baselineskip}
%\titlespacing*{\subsection}{0pt}{\baselineskip}{\baselineskip}

\geometry{left=30mm}
\geometry{right=10mm}
\geometry{top=20mm}
\geometry{bottom=20mm}

\onehalfspacing

\renewcommand{\theenumi}{\arabic{enumi}}
\renewcommand{\labelenumi}{\arabic{enumi}\text{)}}
\renewcommand{\theenumii}{.\arabic{enumii}}
\renewcommand{\labelenumii}{\asbuk{enumii}\text{)}}
\renewcommand{\theenumiii}{.\arabic{enumiii}}
\renewcommand{\labelenumiii}{\arabic{enumi}.\arabic{enumii}.\arabic{enumiii}.}

\renewcommand{\cftchapleader}{\cftdotfill{\cftdotsep}}

\addto\captionsrussian{\renewcommand{\figurename}{Рисунок}}
\DeclareCaptionLabelSeparator{dash}{~---~}
\captionsetup{labelsep=dash}

\captionsetup[figure]{justification=centering,labelsep=dash}
\captionsetup[table]{labelsep=dash,justification=raggedright,singlelinecheck=off}

\graphicspath{{images/}}%путь к рисункам

\newcommand{\floor}[1]{\lfloor #1 \rfloor}

\definecolor{codegreen}{rgb}{0,0.6,0}
\definecolor{codegray}{rgb}{0.5,0.5,0.5}
\definecolor{codepurple}{rgb}{0.58,0,0.82}
\definecolor{backcolour}{rgb}{0.95,0.95,0.92}

\lstdefinestyle{mystyle}{
	backgroundcolor=\color{backcolour},   
	commentstyle=\color{codegreen},
	keywordstyle=\color{magenta},
	numberstyle=\tiny\color{codegray},
	stringstyle=\color{codepurple},
	basicstyle=\ttfamily\footnotesize,
	breakatwhitespace=false,         
	breaklines=true,                 
	captionpos=b,                    
	keepspaces=true,                 
	numbers=left,                    
	numbersep=5pt,                  
	showspaces=false,                
	showstringspaces=false,
	showtabs=false,                  
	tabsize=2
}
\usepackage[russian]{babel} % Поддержка русского языка
\usepackage{listings} % Пакет для вставки кода

\lstset{ % Настройки для листинга
	language=[Sharp]C, % Указываем язык C#
	basicstyle=\ttfamily, % Шрифт для кода
	commentstyle=\color{green!50!black}, % Цвет комментариев
	keywordstyle=\color{blue}, % Цвет ключевых слов
	stringstyle=\color{red}, % Цвет строк
	numbers=left, % Нумерация строк слева
	numberstyle=\tiny\color{gray}, % Стиль нумерации
	stepnumber=1, % Номер строки через 1
	numbersep=5pt, % Расстояние между номером строки и кодом
	frame=single, % Рамка вокруг кода
	captionpos=b, % Позиция заголовка (b - снизу)
	breaklines=true % Перенос длинных строк
}


\lstset{style=mystyle}
\pgfplotsset{width=0.85\linewidth, height=0.5\columnwidth}

\linespread{1.3}

\parindent=1.25cm

%\LetLtxMacro\itemold\item
%\renewcommand{\item}{\itemindent0.75cm\itemold}

\def\labelitemi{---}
\setlist[itemize]{leftmargin=1.25cm, itemindent=0.65cm}
\setlist[enumerate]{leftmargin=1.25cm, itemindent=0.55cm}

\newcommand{\specialcell}[2][c]{%
	\begin{tabular}[#1]{@{}c@{}}#2\end{tabular}}

\frenchspacing