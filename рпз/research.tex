\chapter{Исследовательский раздел}
Было проведено исследование влияния индексов на скорость выполнения запросов при различных объемах данных. На тестовой таблице $batch\_of\_products\_copy$ (имеет такую же структуру как и $batch\_of\_products$) с количеством записей от 10 до 1 миллиона измерялось время выполнения типового запроса с фильтрацией и сортировкой. Замеры проводились в двух режимах: без использования индексов и с оптимизированными индексами (составным для условий фильтрации и отдельным для сортировки).

В листинге~\ref{lst:research} представлен запрос, для которого проводилось исследование и использованные индексы.
\begin{lstlisting}[label=lst:research, caption=Запрос и индексы, language=SQL]
	select id_product_batch, id_nomenclature, production_date, expiration_date
	from batch_of_products_copy
	where 
	production_date between '2023-01-01' and '2023-12-31'
	and cost_price between 100.00 and 500.00
	and amount > 50
	order by expiration_date desc;
	
	create index idx_composite_filter on batch_of_products_copy(production_date, cost_price, amount);
	create index idx_sorting on batch_of_products_copy(expiration_date desc);
\end{lstlisting}

\begin{table}[h]
	\centering
	\caption{Сравнение времени выполнения запросов с индексами и без (в секундах)}
	\label{tab:query_performance}
	\begin{tabular}{|p{3cm}|p{3cm}|p{4cm}|}
		\hline
		\textbf{Количество записей} & \textbf{Без индекса} & \textbf{С индексом} \\
		\hline
		10 & 0.000040 & 0.000008 \\
		\hline
		100 & 0.000051 & 0.000014 \\
		\hline
		1,000 & 0.000146 & 0.000055 \\
		\hline
		10,000 & 0.002802 & 0.000699 \\
		\hline
		100,000 & 0.030805 & 0.017392 \\
		\hline
		1,000,000 & 0.129758 & 0.128987 \\
		\hline
	\end{tabular}
\end{table}